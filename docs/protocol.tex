\chapter{Protocols}
\section{Introduction}

The OpenAMD protocols have a short stack of containers:
there's an outer, packetizing link layer transport
({\it e.g.}, nRF24 protocol, UDP/IP)
and an inner TLV (Type-Length-Value) stream.
We require that TLV entries fit inside packet boundaries.

\section{Link Layer Transports}

\subsection{nRF}

The nRF link layer is a little funny.
The documentation does not describe
how to put chips into promiscuous mode,
and the host-facing SPI transport does not include
the destination MAC address of packets it receives,
only the ``MultiCeiver\texttrademark~ channel''.
MultiCeiver\texttrademark~ channels are configurable oddly:
channel 0 has its own address,
and channels 1 through 5 have their prefix bytes in common.
There is no such thing as a source address on the air.
We imagine that the intended use of this addressing scheme was to
allow for a singular unicast address and many multicasts all using
a common prefix, but it still strikes us as a little odd.

In addition to knowing which MAC a given network is using,
one has to know its channel (frequency)
and the flavor of checksumming used on the radio to filter out noise, if any.

OpenAMD uses the MAC to distinguish protocols (TLV streams) being encapsulated.
Currently, OpenAMD defines the channel/MAC pairs shown in \autoref{fig:proto:nrfmac}.

\begin{figure}[p]
    \begin{center}\begin{tabular}{lllll}
        Channel & MAC                       & CRC     & Protocol & XREF \\
        81      & 0x424541434F (``BEACO'')  & 1 byte
            & OpenBeacon Beacon Container & \autoref{sec:proto:openbeacon} \\
        81      & 0x0102030201              & 1 byte
            & OpenBeacon OpenBouncer      & \autoref{sec:proto:openbouncer} \\
    \end{tabular}\end{center}
    \caption{Reserved OpenAMD nRF encapsulation schemes}
    \label{fig:proto:nrfmac}
\end{figure}

\subsection{UDP/IP}

This is pretty bog standard.
We strip off the nRF container
and plop the data into a UDP packet.
The source address of the packet
is the node doing the decapsulation.
For OpenBeacon TLV streams,
we adopt use of UDP port 2342.

\section{OpenBeacon TLV}

This framing scheme is stolen from OpenBeacon \cite{openbeacon}.
We believe this is the first attempt to describe it formally
and centrally collect the types used internally.

OpenBeacon TLV packets
are framed by a two byte header,
consisting of the length and type byes,
and are terminated with a CRC-16 checksum.
The length byte includes itself,
the type byte,
and the two-byte checksum.
See \autoref{fig:proto:obframe}.
All multi-byte fields 
are in Network Endian order
({\it i.e.} MSB first).
Type values are assigned as in \autoref{fig:proto:obtypes}

\begin{figure}[p]
    \begin{center}\begin{bytefield}{16}
        \bitheader{0,7-8,15}\\
        \bitbox{8}{Length} & \bitbox{8}{Type} \\
        \skippedwords \\
        \bitbox{16}{CRC-16}
    \end{bytefield}\end{center}
    \caption{OpenBeacon TLV Framing}
    \label{fig:proto:obframe}
\end{figure}

\begin{figure}[p]
    \begin{center}\begin{tabular}{ll}
        Type & Payload \\
        {\tt 0x17} & OpenBeacon Tracker Report \\
    \end{tabular}\end{center}
    \caption{OpenBeacon TLV Types}
    \label{fig:proto:obtypes}
\end{figure}

\subsection{Backwards Compatibility with OpenBeacon Data}

Unfortunately OpenBeacon was not consistent
in their assignments of type fields.
At various points in the past,
the following types have been in use
for one or more protocols.
We list parenthetically the mnemonics
used in various parts of the source tree
for each type identifier.
\begin{itemize}
    \item {\tt 0x16} ({\tt RFBPROTO\_READER\_ANNOUNCE})
    \item {\tt 0x17} ({\tt RFBPROTO\_READER\_COMMAND}, {\tt RFBPROTO\_BEACONTRACKER})
    \item {\tt 0x18} ({\tt RFBPROTO\_BEACONTRACKER}, {\tt RFBPROTO\_BEACONNODE})
    \item {\tt 0x19} ({\tt RFBPROTO\_BEACONVOTESTATS})
    \item {\tt 0x2A} ({\tt RFBPROTO\_PROXTRACKER}, {\tt RFBPROTO\_BLINKENLIGHTS})
    \item {\tt 0x45} ({\tt RFBPROTO\_PROXREPORT})
\end{itemize}
Further, OpenBeacon was not consistent in layout of data
even within type mnemonic.
For example, there are two different
structures implied by {\tt RFBPROTO\_BEACONTRACKER}
(one for value {\tt 0x17}, and one for {\tt 0x18}).

This unfortunate result probably came about by lack of
a central registry of identifiers and structures.
For those eager to decode old badge beacons,
we {\em believe} that
packet types other than {\tt RFBPROTO\_BEACONTRACKER}
are rare in the wild.
See \autoref{sec:proto:openbeacontrackback}

\subsection{OpenBeacon Tracker Report}
\label{sec:proto:openbeacon}

This protocol is stolen from the OpenBeacon project \cite{openbeacon}.

The data inside a Tracker Report is shown in \autoref{fig:proto:obtrp}.
(The length field for this kind of packet is always {\tt 0x10}.)

The {\tt SENSE} field is the result of reading on-board sensors.
% TODO

The transmission strength ({\tt TX STR}) data is
an indicator of how loudly the packet was sent;
since the nRF chips do not
give an indication of RSSI,
the beacons are expected
to vary their transmission power with time
to provide the receivers
with some approximate idea
of link quality.


\begin{figure}[p]
    \begin{center}\begin{bytefield}{16}
        \bitheader{0,7-8,15}\\
        \bitbox{8}{\tt SENSE} & \bitbox{8}{\tt TX STR} \\
        \bitbox{16}{\tt Seq Hi} \\
        \bitbox{16}{\tt Seq Lo} \\
        \bitbox{16}{\tt Device ID Hi} \\
        \bitbox{16}{\tt Device ID Lo} \\
        \bitbox{16}{{\tt RESERVED (0x00)}} \\
    \end{bytefield}\end{center}
    \caption{OpenBeacon Tracker Report Payload}
    \label{fig:proto:obtrp}
\end{figure}

\subsubsection{Backwards Compatibility with OpenBeacon Tracker Reports}
\label{sec:proto:openbeacontrackback}

However, OpenBeacon, for unknown reasons,
chooses to encrypt the entire network with XXTEA
\cite{needham:xtea,wheeler:xxtea}
using a single, static, published key per event.
As an extremely ugly special case,
in the spirit of hardware reuse,
we (unfortunately) recommend
attempting to decode,
using one or more of these keys
(republished in \autoref{fig:proto:xxteakeys}
as an act of neighborly good will),
OpenBeacon TLV streams which fail to frame,
and, if the result frames properly,
parse that instead.

The data layout chosen above corresponds with
OpenBeacon's data layout exactly when
{\tt RFBPROTO\_BEACONTRACKER} had value {\tt 0x17}.
The other packet format, corresponding to {\tt 0x18},
is described in \autoref{fig:proto:oboldtracker}
solely for the purpose of backwards compatibility.
Attempts to frame packets of this shape must only
be made when attempting to fall back and XXTEA
decryption has been applied.

Further, note that the field labeled {\tt SENSE}
in \autoref{fig:proto:obtrp} was, in OpenBeacon firmwares,
a collection of flags
which varied by firmware.
% TODO

\begin{figure}[p]
    \begin{center} \begin{tabular}{cl}
        Key & Used At \\
        {\tt 0x7013F569,0x4417CA7E,0x07AAA968,0x822D7554} & 25C3 free beta version key \\
        {\tt 0x8e7d6649,0x7e82fa5b,0xddd4541e,0xe23742cb} & Camp 2007 key \\
        {\tt 0x9c43725e,0xad8ec2ab,0x6ebad8db,0xf29c3638} & The Last Hope - AMD \\
        {\tt 0xb4595344,0xd3e119b6,0xa814d0ec,0xeff5a24e} & 24C3 key \\
        {\tt 0xbf0c3a08,0x1d4228fc,0x4244b2b0,0x0b4492e9} & 25C3 final key \\
        {\tt 0xdeadbeef,0xdeadbeef,0xdeadbeef,0xdeadbeef} & 23C3 ? \\
        {\tt 0xe107341e,0xab99c57e,0x48e17803,0x52fb4d16} & 23C3 key \\
        {\tt 0xee2522d1,0xdbc221f1,0xa21d0d0e,0x865976a2} & ? \\
    \end{tabular} \end{center}
    \caption{XXTEA keys used by OpenBeacon installations at times past.}
    \label{fig:proto:xxteakeys}
\end{figure}

\begin{figure}[p]
    \begin{center}\begin{bytefield}{16}
        \bitheader{0,7-8,15}\\
        \bitbox{8}{\tt TX STR} & \bitbox{8}{\tt Device ID Hi} \\
        \bitbox{8}{\tt Device ID Lo} & \bitbox{8}{\tt Boot Count Hi} \\
        \bitbox{8}{\tt Boot Count Lo} & \bitbox{8}{\tt RESERVED} \\
        \bitbox{16}{\tt Sequence Count Hi} \\
        \bitbox{16}{\tt Sequence Count Lo} \\
    \end{bytefield}\end{center}
    \caption{Obsolete OpenBeacon Tracker Report Payload}
    \label{fig:proto:oboldtracker}
\end{figure}


\subsection{OpenBeacon OpenBouncer Protocol}
\label{sec:proto:openbouncer}

The OpenBeacon subproject OpenBouncer uses
a different format on the air altogether
and is not documented here
save to reserve its nRF MAC address.
